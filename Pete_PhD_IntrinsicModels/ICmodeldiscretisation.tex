\documentclass{article}

\usepackage{amsmath}
\usepackage{IEEEtrantools}
\usepackage{color}

\usepackage{mdwmath}
\usepackage{mdwtab}

\newenvironment{meta}[0]{\color{red} \em}{}

%opening
\title{Discretisation of Intrinsic Coordinate Dynamic Models}
\author{Pete Bunch}

\begin{document}
\maketitle

\section{Differential Equations of Motion in Intrinsic Coordinates}

Define position, $\mathbf{r}(t)$, velocity, $\mathbf{v}(t)$, and acceleration, $\mathbf{a}(t)$ and a set of unit vectors tangential, normal, and binormal to the motion,
%
\begin{IEEEeqnarray}{rCl}
 \mathbf{e}_{T}(t) &\propto& \mathbf{v}(t) \\
 \mathbf{e}_{B}(t) &\propto& \mathbf{v}(t) \times \mathbf{a}(t) \\
 \mathbf{e}_{N}(t) & =     & \mathbf{e}_{B}(t) \times \mathbf{e}_{T}(t)     .
\end{IEEEeqnarray}

We'll also need the speed,
%
\begin{IEEEeqnarray}{rCl}
 \dot{s}(t) &=& |\mathbf{v}(t)|     .
\end{IEEEeqnarray}

Begin by looking at how the unit vectors differentiate,
%
\begin{IEEEeqnarray}{rCl}
 \dot{\mathbf{e}}_{T}(t) & = & \dot{\psi}(t) \mathbf{e}_{N}(t) \\
 \dot{\mathbf{e}}_{B}(t) & = & \dot{\phi}(t) \mathbf{e}_{N}(t) \\
 \dot{\mathbf{e}}_{N}(t) & = & - \dot{\psi}(t) \mathbf{e}_{T}(t) - \dot{\phi}(t) \mathbf{e}_{B}(t)
\end{IEEEeqnarray}

\noindent where $\dot{\psi}(t)$ is the instantaneous rate of turn, and $\dot{\phi}(t)$ is the instantaneous rate of roll.

Now we can write the acceleration as,
%
\begin{IEEEeqnarray}{rCl}
\mathbf{a}(t) &=& \dot{\mathbf{v}}(t) \nonumber \\
              &=& \frac{d}{dt}\left( \dot{s}(t) \mathbf{e}_{T}(t) \right) \nonumber \\
              &=& \ddot{s}(t) \mathbf{e}_{T}(t) + \dot{s}(t) \dot{\mathbf{e}}_{T}(t) \nonumber \\
              &=& \underbrace{\ddot{s}(t)}_{a_T} \mathbf{e}_{T}(t) + \underbrace{\dot{s}(t) \dot{\psi}(t)}_{a_N} \mathbf{e}_{N}(t)     ,
\end{IEEEeqnarray}

The principle upon which the discretisation is based is that $a_T$ and $a_N$ are constant for the duration of a ``manoeuvre'',
%
\begin{IEEEeqnarray}{rCl}
 \ddot{s}(t) &=& a_T \nonumber \\
 \dot{s}(t) \dot{\psi}(t) &=& a_N     .
\end{IEEEeqnarray}

Alone, this is not sufficient to render the model analytically solvable, but lets see how far we can get. First the speed,
%
\begin{IEEEeqnarray}{rCl}
 \dot{s}(t) &=& s_0 + a_T t     ,
\end{IEEEeqnarray}

\noindent where $s_0 = s(0)$ is the fixed intial value. Second the rotation,
%
\begin{IEEEeqnarray}{rCl}
 \dot{\psi}(t) &=& \frac{a_N}{ \dot{s}(t) } \nonumber \\
               &=& \frac{a_N}{ s_0 + a_T t } \nonumber \\
 \Delta \psi(t) &=& \int_0^t \dot{\psi}(\tau) d\tau \nonumber \\
                &=& \frac{a_N}{a_T} \log \left[ \frac{s(t)}{s_0} \right]     .
\end{IEEEeqnarray}

The variable $\Delta \psi(t)$ is the total rotation since the start of the manoeuvre. In general, there is no reason for this rotation to lie within a plane, so $\Delta \psi(t)$ is \emph{not} the same as the angle between $\mathbf{e}_{T}(0)$ and $\mathbf{e}_{T}(t)$.

Now we group the unit vector differential equations together into a matrix equation,
%
\begin{IEEEeqnarray}{rCl}
\underbrace{\frac{d}{dt}\begin{bmatrix}\mathbf{e}_T(t)^T \\ \mathbf{e}_N(t)^T \\ \mathbf{e}_B(t)^T \end{bmatrix}}_{\dot{E}(t)} &=& \underbrace{\begin{bmatrix}0 & \dot{\psi}(t) & 0 \\ - \dot{\psi}(t) & 0 & - \dot{\phi}(t) \\ 0 & \dot{\phi}(t) & 0 \end{bmatrix}}_{F(t)} \underbrace{\begin{bmatrix}\mathbf{e}_{T}^T(t) \\ \mathbf{e}_{N}^T(t) \\ \mathbf{e}_{B}^T(t) \end{bmatrix}}_{E(t)} \label{eq:matrixdiffeq}     .
\end{IEEEeqnarray}

The roll rate, $\dot{\phi}(t)$ is still not specified by the assumptions we've made so far. In general, this non-homogeneous matrix differential equation has no solution. However, we can find solutions for particular cases.



\section{Solutions to Matrix Differential Equations}

It is well known that the matrix exponential solves the matrix differential equation (\ref{eq:matrixdiffeq}) when $F(t)$ is constant with respect to time. In fact the class of problems for which this solution holds is larger. Consider the time derivative of the matrix exponential,
%
\begin{IEEEeqnarray}{rCl}
 X(t)       & = & \exp \left\{ Q(t) \right\} = \sum_{n=0}^{\infty} \frac{1}{n!} Q(t)^n \nonumber \\
 \dot{X}(t) & = & \sum_{n=1}^{\infty} \frac{1}{n!} \sum_{i=1}^{n} Q(t)^{n-i} \dot{Q}(t) Q(t)^{i-1} \nonumber \\
            & = & \sum_{n=1}^{\infty} \frac{1}{n!} \underbrace{\sum_{i=1}^{n} \dot{Q}(t) Q(t)^{n-1}}_{n \dot{Q}(t) Q(t)^{n-1}} \label{eq:commute_with_deriv} \\
            & = & \dot{Q}(t) \sum_{m=0}^{\infty} \frac{1}{m!} Q(t)^{m} \nonumber \\
            & = & \dot{Q}(t) \exp\left\{ Q(t) \right\} \nonumber     .
\end{IEEEeqnarray}

Thus, the solution to (\ref{eq:matrixdiffeq}) follows by equating $\dot{Q}(t)$ and $F(t)$ and using the initial condition,
%
\begin{IEEEeqnarray}{rCl}
 E(t) &=& \exp \left( \int_0^t F(\tau) d\tau + K \right) \nonumber \\
      &=& \underbrace{\exp \left( \int_0^t F(\tau) d\tau \right)}_{A(t)} E_0    .
\end{IEEEeqnarray}

In (\ref{eq:commute_with_deriv}), the condition required is that $Q(t)$ should commute with its derivative, i.e. $Q(t)\dot{Q}(t)=\dot{Q}(t)Q(t)$. From this it is straightforward to show that,
%
\begin{IEEEeqnarray}{rCl}
 Q(t)\dot{Q}(t) & = & \dot{Q}(t)Q(t) \nonumber \\
 \int_0^t F(\tau) d\tau F(t) & = & F(t) \int_0^t F(\tau) d\tau \nonumber \\
 \int_0^t F(\tau) F(t) d\tau & = & \int_0^t F(t) F(\tau) d\tau \nonumber \\
 F(\tau) F(t) & = & F(t) F(\tau)     .
\end{IEEEeqnarray}
%
%\begin{IEEEeqnarray}{rCl}
% Q(t)\dot{Q}(t) & = & \dot{Q}(t)Q(t) \nonumber \\
% Q(t) \lim_{h\rightarrow0} \left[ \frac{Q(t+h)-Q(t)}{h} \right] & = &  \lim_{h\rightarrow0} \left[ \frac{Q(t+h)-Q(t)}{h} \right] Q(t) \nonumber \\
% \lim_{h\rightarrow0} \left[ \frac{Q(t)Q(t+h)-Q(t)^2}{h} \right] & = &  \lim_{h\rightarrow0} \left[ \frac{Q(t+h)Q(t)-Q(t)^2}{h} \right] \nonumber \\
% Q(t)Q(t+h) & = & Q(t+h)Q(t)     .
%\end{IEEEeqnarray}

So the matrix exponential solves the matrix differential equation when $F(t)$ commutes with itself with a time offset.

{\meta Is this necessary and sufficient? Or just sufficient? Ask a mathematician.}



\section{Planar Motion}

The simplest solution results from the assumption $\dot{\phi}(t) = 0$. In this case,
%
\begin{IEEEeqnarray}{rCl}
 \Delta \phi(t) &=& 0 \nonumber \\
 F(t) & = & \begin{bmatrix}0 & \dot{\psi}(t) & 0 \\ - \dot{\psi}(t) & 0 & 0 \\ 0 & 0 & 0 \end{bmatrix}
\end{IEEEeqnarray}

Testing the sufficient condition,
%
\begin{IEEEeqnarray}{rCl}
 F(t_1)F(t_2) = F(t_2) F(t_1) = \begin{bmatrix} -\dot{\psi}(t_1)\dot{\psi}(t_2) & 0 & 0 \\ 0 & -\dot{\psi}(t_1)\dot{\psi}(t_2) & 0 \\ 0 & 0 & 0 \end{bmatrix}     .
\end{IEEEeqnarray}

Hence the equation is solved by,
%
\begin{IEEEeqnarray}{rCl}
 E(t) & = & A(t) E_0 \\
 A(t) & = & \exp \left\{ \int_0^t F(\tau) d\tau \right\}
\end{IEEEeqnarray}

This may be solved by diagonalisation.
%
\begin{IEEEeqnarray}{rCl}
 F(t) & = & \begin{bmatrix} \frac{i}{\sqrt{2}} & -\frac{i}{\sqrt{2}} & 0 \\ \frac{1}{\sqrt{2}} & \frac{1}{\sqrt{2}} & 0 \\ 0 & 0 & 1 \end{bmatrix}  \begin{bmatrix} -\frac{a_N i}{s(t)} & 0 & 0 \\ 0 & \frac{a_N i}{s(t)} & 0 \\ 0 & 0 & 0 \end{bmatrix} \begin{bmatrix} \frac{i}{\sqrt{2}} & \frac{1}{\sqrt{2}} & 0 \\ \frac{-i}{\sqrt{2}} & \frac{1}{\sqrt{2}} & 0 \\ 0 & 0 & 1 \end{bmatrix}
\end{IEEEeqnarray}

After some tedious algebra, the matrix exponential is given by,
%
\begin{IEEEeqnarray}{rCl}
 A(t) &=& \begin{bmatrix} \cos(\Delta \psi(t)) & \sin(\Delta \psi(t)) & 0 \\
                         -\sin(\Delta \psi(t)) & \cos(\Delta \psi(t)) & 0 \\
                          0                    & 0                    & 0 \end{bmatrix}
\end{IEEEeqnarray}

This describes motion within the plane spanned by $\mathbf{e}_{T}(0)$ and $\mathbf{e}_{N}(0)$. If $\mathbf{e}_{B}(0) = \mathbf{k}$, then this reduces to the 2D model, in which velocity may be expressed in terms of speed and heading.

The velocity and position are given by,
%
\begin{IEEEeqnarray}{rCl}
 \mathbf{v}(t)^T & = & \dot{s}(t) \mathbf{e}_{T}(t)^T \nonumber \\
                 & = & \dot{s}(t) \begin{bmatrix} 1 & 0 & 0 \end{bmatrix} E(t) \\
                 & = & \begin{bmatrix} 1 & 0 & 0 \end{bmatrix} \dot{s}(t) A(t) E_0 \\
 \mathbf{r}(t)^T & = & \int_0^t \mathbf{v}(\tau)^T d\tau \nonumber \\
                 & = & \begin{bmatrix} 1 & 0 & 0 \end{bmatrix} \int_0^t \dot{s}(\tau) A(\tau) d\tau E_0 \nonumber \\
                 & = & \frac{1}{a_N^2+ 4 a_T^2} \begin{bmatrix} 2 a_T & a_N & 0 \end{bmatrix} \begin{bmatrix} \zeta_1(t) & \zeta_2(t) & 0 \\ \zeta_2(t) & -\zeta_1(t) & 0 \\ 0 & 0 & 0 \end{bmatrix} E_0 \nonumber \\
 \zeta_1(t)      & = & \cos(\Delta \psi(t))\dot{s}(t)-\dot{s}_0 \nonumber \\
 \zeta_2(t)      & = & \sin(\Delta \psi(t))\dot{s}(t) \nonumber     .
\end{IEEEeqnarray}

The unpleasant intermediate integration and factorisation used here is,
%
{\tiny
\begin{IEEEeqnarray}{rCl}
 \int_0^t \dot{s}(\tau) A(\tau) d\tau & = &\begin{bmatrix} \frac{1}{a_N^2+ 4 a_T^2} \left\{ 2 a_T \left[\cos(\Delta \psi(t))\dot{s}(t)-\dot{s}_0\right] + a_N \sin(\Delta \psi(t))\dot{s}(t) \right\} & \frac{1}{a_N^2+ 4 a_T^2} \left\{ -a_N \left[\cos(\Delta \psi(t))\dot{s}(t)-\dot{s}_0\right] + 2 a_T \sin(\Delta \psi(t))\dot{s}(t) \right\} & 0 \\ \frac{1}{a_N^2+ 4 a_T^2} \left\{ a_N \left[\cos(\Delta \psi(t))\dot{s}(t)-\dot{s}_0\right] - 2 a_T \sin(\Delta \psi(t))\dot{s}(t) \right\} & \frac{1}{a_N^2+ 4 a_T^2} \left\{ 2 a_T \left[\cos(\Delta \psi(t))\dot{s}(t)-\dot{s}_0\right] + a_N \sin(\Delta \psi(t))\dot{s}(t) \right\} & 0 \\ 0 & 0 & 0 \end{bmatrix} \nonumber \\
 & = & \frac{1}{a_N^2+ 4 a_T^2} \begin{bmatrix} 2 a_T & a_N & 0 \\ aN & -2 a_T & 0 \\ 0 & 0 & 0 \end{bmatrix} \begin{bmatrix} \cos(\Delta \psi(t))\dot{s}(t)-\dot{s}_0 & \sin(\Delta \psi(t))\dot{s}(t) & 0 \\ \sin(\Delta \psi(t))\dot{s}(t) & -(\cos(\Delta \psi(t))\dot{s}(t)-\dot{s}_0) & 0 \\ 0 & 0 & 0 \end{bmatrix} \nonumber     .
\end{IEEEeqnarray}
}



\section{Constant Roll Rate}

At one point I thought, erroneously, that it was possible to extend the model by assuming the roll rate to be constant,
%
\begin{IEEEeqnarray}{rCl}
 \Delta \phi(t) &=& r_R t     .
\end{IEEEeqnarray}

However, I think this actually renders (\ref{eq:matrixdiffeq}) unsolvable, so we're stuffed.

The most promising approach appears to be transforming $F(t)$, using an eigen-decomposition or the skew-symmetric decomposition,
%
\begin{IEEEeqnarray}{rCl}
 F(t) & = & U(t) \Sigma(t) U(t)^T \\
 U(t) & = & \begin{bmatrix} \frac{\dot{\psi}(t)}{h(t)} & 0 & \frac{-\dot{\phi}(t)}{h(t)} \\ 0 & 1 & 0 \\ \frac{\dot{\phi}(t)}{h(t)} & 0 & \frac{\dot{\psi}(t)}{h(t)} \end{bmatrix} \\
 \Sigma(t) & = & \begin{bmatrix} 0 & h(t) & 0 \\ -h(t) & 0 & 0 \\ 0 & 0 & 0 \end{bmatrix} \\
 h(t) & = & \sqrt{\dot{\psi}^2 + \dot{\phi}^2}     .
\end{IEEEeqnarray}



\section{Proportional Roll Rate}

We can get an analytic solution if we assume the following form for the roll rate,
%
\begin{IEEEeqnarray}{rCl}
 \Delta \phi(t) & = & \frac{a_R}{\dot{s}(t)}     .
\end{IEEEeqnarray}

This allows the matrix in the differential equation to be factorised as,
%
\begin{IEEEeqnarray}{rCl}
 F(t) & = & \frac{1}{\dot{s}(t)} \underbrace{\begin{bmatrix} 0 & a_N & 0 \\ -a_N & 0 & -a_R \\ 0 & a_R & 0 \end{bmatrix}}_{\bar{F}}     .
\end{IEEEeqnarray}

This (obviously) satisfies our condition for a matrix exponential solution. Hence the equation is solved by,
%
\begin{IEEEeqnarray}{rCl}
 E(t) & = & A(t) E_0 \\
 A(t) & = & \exp \left\{ \int_0^t F(\tau) d\tau \right\} \nonumber \\
      & = & \exp \left\{ \bar{F} \int_0^t \frac{1}{\dot{s}(t)} d\tau \right\}
\end{IEEEeqnarray}

This matrix exponential may be found analytically using a modified form of the theorem of \cite{Bernstein1993} (section IV). See appendix~\ref{app:ssmatrixexp}.
%
\begin{IEEEeqnarray}{rCl}
 A(t) & = & \exp \left\{ \bar{F} \vartheta(t) \right\} \nonumber \\
      & = & I + \frac{\bar{F}}{a_C} \sin(a_C \vartheta(t)) - \frac{\bar{F}^2}{a_c^2} \left[ \cos(a_C \vartheta(t)) - 1 \right] \\
 a_C          & = & \sqrt{a_N^2 + a_R^2} \\
 \vartheta(t) & = & \int_0^t \frac{1}{\dot{s}(t)} d\tau \nonumber \\
              & = & \frac{1}{a_T} \log \left[ \frac{\dot{s}(t)}{\dot{s}_0} \right]
\end{IEEEeqnarray}

The velocity and position are given by,
%
\begin{IEEEeqnarray}{rCl}
 \mathbf{v}(t)^T & = & \dot{s}(t) \mathbf{e}_{T}(t)^T \nonumber \\
                 & = & \dot{s}(t) \begin{bmatrix} 1 & 0 & 0 \end{bmatrix} E(t) \\
                 & = & \begin{bmatrix} 1 & 0 & 0 \end{bmatrix} \dot{s}(t) A(t) E_0 \\
 \mathbf{r}(t)^T & = & \int_0^t \mathbf{v}(\tau)^T d\tau \nonumber \\
                 & = & \begin{bmatrix} 1 & 0 & 0 \end{bmatrix} \int_0^t \dot{s}(\tau) A(\tau) d\tau E_0 \nonumber \\
 \int_0^t \dot{s}(\tau) A(\tau) d\tau & = & \left[ I + \frac{\bar{F}^2}{a_C^2} \right] \left( s_0 t + \frac{1}{2} a_T t^2 \right) \nonumber \\
                                      &   & + \: \frac{\bar{F}}{a_C(a_C^2+4a_T^2)} \left[ 2a_T \zeta_2(t) - a_C \zeta_1(t) \right] \\
                                      &   & - \: \frac{\bar{F}^2}{a_C^2(a_C^2+4a_T^2)} \left[ a_C \zeta_2(t) + 2a_T \zeta_1(t) \right] \\
 \zeta_1(t)      & = & \cos(a_C \vartheta(t))\dot{s}(t)-\dot{s}_0 \nonumber \\
 \zeta_2(t)      & = & \sin(a_C \vartheta(t))\dot{s}(t) \nonumber     .
\end{IEEEeqnarray}

{\meta Check this integration.}

Thus, at last, we have everything we need to calculate $\mathbf{r}(t)$ and $\mathbf{v}(t)$ given initial values (and an initial roll --- see next section). This gives us an analytic solution for a type of non-planar trajectory, based upon an additional assumption of the form of the roll rate. The question is, whether it is a realistic assumption. It may not be sensible at all.

The planar model may be recovered as a special case by setting $a_R = 0$ throughout.


\section{Initialisation}

Assuming we know the initial speed and velocity, we still need a procedure for specifying the initial normal and binormal unit vectors. This can be done by introducing a new parameter, $\phi_0$, to indicate the initial angle between the binormal vector and the vertical plane containing the velocity vector.

\begin{tabular}{lm{4cm}}
\renewcommand{\arraystretch}{1.5}
\\
Perpendicular to the                    & $\mathbf{e}_{T,0} \cdot \mathbf{e}_{B,0} = 0$ \\
tangential vector:                      &  \\
Unit magnitude:                         & $\left| \mathbf{e}_{B,0} \right| = 1$         \\
Angle $\phi_{0}$ to the                 &  \\
vertical plane:                         & $\mathbf{e}_{B,0} \cdot \frac{\mathbf{e}_{T,0} \times \mathbf{k}}{\left|\mathbf{e}_{T,0} \times \mathbf{k}\right|} = \sin(\phi_{0})$ \\ \\
\end{tabular}

\noindent where $\mathbf{k}$ is the $z$ axis unit vector. Solving these, it may be shown that if $\mathbf{e}_{T,0} = [e_{T1,0}, e_{T2,0}, e_{T3,0}]^T$, then the binormal unit vector is given by,
%
\begin{IEEEeqnarray}{rCl}
 \mathbf{e}_{B,k} = \begin{bmatrix}
                    \frac{e_{T2,0} \sin(\phi_0) - e_{T1,0} e_{T3,0} \cos(\phi_0)}{\sqrt{e_{T1,0}^2+e_{T2,0}^2}} \\
                    \frac{-e_{T1,0} \sin(\phi_0) - e_{T2,0} e_{T3,0} \cos(\phi_0)}{\sqrt{e_{T1,0}^2+e_{T2,0}^2}} \\
                    \cos(\phi_0) \sqrt{e_{T1,0}^2+e_{T2,0}^2}
                \end{bmatrix}     .
\end{IEEEeqnarray}

Finally, the normal unit vector is given by,
%
\begin{IEEEeqnarray}{rCl}
 \mathbf{e}_{N,0} = \mathbf{e}_{B,0} \times \mathbf{e}_{T,0}
\end{IEEEeqnarray}



\section{Gravity}

It would be nice to include gravity, but it does not fit easily into the framework. It can be linearly approximated over short time by just adding a gravity term to the velocity and position equations.



\appendix

\section{Matrix Exponential of a class of Skew-Symmetric Matrices} \label{app:ssmatrixexp}

Consider a matrix of the form,
%
\begin{IEEEeqnarray}{rCl}
 A & = & \begin{bmatrix} 0 & a & 0 \\ -a & 0 & -b \\ 0 & b & 0 \end{bmatrix}     .
\end{IEEEeqnarray}

It can be shown that for $n>2$,
%
\begin{IEEEeqnarray}{rCl}
 A^n & = & -(a^2+b^2) A^{n-2}     .
\end{IEEEeqnarray}

Proof is by induction. In both the following cases,
%
\begin{IEEEeqnarray}{rCl}
 A^{n-2} & = & \begin{bmatrix} 0 & a_{n-2} & 0 \\ -a_{n-2} & 0 & -b_{n-2} \\ 0 & b_{n-2} & 0 \end{bmatrix} \nonumber \\
 A^{n-2} & = & \begin{bmatrix} c_{n-2} & 0 & d_{n-2} \\ 0 & e_{n-2} & 0 \\ d_{n-2} & 0 & f_{n-2} \end{bmatrix} \nonumber     ,
\end{IEEEeqnarray}

\noindent then by calculation,
%
\begin{IEEEeqnarray}{rCl}
 A^n & = & A^2 A^{n-2} = -(a^2+b^2) A^{n-2}     .
\end{IEEEeqnarray}

The first two powers are given by,
%
\begin{IEEEeqnarray}{rCl}
 A^1 & = & \begin{bmatrix} 0 & a & 0 \\ -a & 0 & -b \\ 0 & b & 0 \end{bmatrix} \nonumber \\
 A^2 & = & \begin{bmatrix} -a^2 & 0 & -ab \\ 0 & -(a^2+b^2) & 0 \\ -ab & 0 & -b^2 \end{bmatrix}     .
\end{IEEEeqnarray}

Thus we can write any power of $A$ as,
%
\begin{IEEEeqnarray}{rCl}
 A^n & = & \begin{cases} \left[ -(a^2+b^2) \right]^{\frac{n-2}{2}} A^2 & n \text{ even} \\
                         \left[ -(a^2+b^2) \right]^{\frac{n-1}{2}} A & n \text{ odd} \end{cases}
\end{IEEEeqnarray}

Now consider the matrix exponential of such a matrix multiplied by a scalar function,
%
\begin{IEEEeqnarray}{rCl}
 \IEEEeqnarraymulticol{3}{l}{ \exp\left\{ A x(t) \right\} = \sum_{n=0}^{\infty} \frac{1}{n!} (A x(t))^n } \nonumber \\
 & = & \sum_{n=0}^{\infty} \frac{1}{n!} A^n x(t)^n \nonumber \\
 & = & I + \sum_{n=1 : n \text{ odd}}^{\infty} \frac{1}{n!} \left[ -(a^2+b^2) \right]^{\frac{n-1}{2}} A x(t)^n + \sum_{n=1 : n \text{ even}}^{\infty} \frac{1}{n!} \left[ -(a^2+b^2) \right]^{\frac{n-2}{2}} A^2 x(t)^n \nonumber \\
 & = & I      + \frac{A}{\sqrt{a^2+b^2}} \sum_{n=1 : n \text{ odd}}^{\infty} (-1)^{\frac{n-1}{2}} \frac{1}{n!} \left[ \sqrt{a^2+b^2} x(t) \right]^{n}      - \frac{A^2}{a^2+b^2} \sum_{n=1 : n \text{ even}}^{\infty} (-1)^{\frac{n}{2}} \frac{1}{n!} \left[ \sqrt{a^2+b^2} x(t) \right]^{n} \nonumber \\
 & = & I      + \frac{A}{\sqrt{a^2+b^2}} \sin(\sqrt{a^2+b^2} x(t))      - \left[ \cos(\sqrt{a^2+b^2} x(t)) - 1 \right] 
\end{IEEEeqnarray}


\bibliographystyle{IEEEtran}
\bibliography{D:/pb404/Dropbox/PhD/OTbib}
\end{document}
