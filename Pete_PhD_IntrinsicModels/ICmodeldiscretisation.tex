\documentclass{article}

\usepackage{amsmath}
\usepackage{IEEEtrantools}

%opening
\title{Discretisation of Intrinsic Coordinate Dynamic Models}
\author{Pete Bunch}

\begin{document}
\maketitle

\section{Differential Equations of Motion in Intrinsic Coordinates}

Define position, $\mathbf{r}(t)$, velocity, $\mathbf{v}(t)$, and acceleration, $\mathbf{a}(t)$ and a set of unit vectors tangential, normal, and binormal to the motion,
%
\begin{IEEEeqnarray}{rCl}
 \mathbf{e}_{T}(t) &\propto& \mathbf{v}(t) \\
 \mathbf{e}_{B}(t) &\propto& \mathbf{v}(t) \times \mathbf{a}(t) \\
 \mathbf{e}_{N}(t) & =     & \mathbf{e}_{B}(t) \times \mathbf{e}_{T}(t)     .
\end{IEEEeqnarray}

We'll also need the speed,
%
\begin{IEEEeqnarray}{rCl}
 \dot{s}(t) &=& |\mathbf{v}(t)|     .
\end{IEEEeqnarray}

Begin by looking at how the unit vectors differentiate,
%
\begin{IEEEeqnarray}{rCl}
 \dot{\mathbf{e}}_{T}(t) & = & \dot{\psi}(t) \mathbf{e}_{N}(t) \\
 \dot{\mathbf{e}}_{B}(t) & = & \dot{\phi}(t) \mathbf{e}_{N}(t) \\
 \dot{\mathbf{e}}_{N}(t) & = & - \dot{\psi}(t) \mathbf{e}_{T}(t) - \dot{\phi}(t) \mathbf{e}_{B}(t) \\
\end{IEEEeqnarray}

\noindent where $\dot{\psi}(t)$ is the instantaneous rate of turn, and $\dot{\phi}(t)$ is the instantaneous rate of roll.

Now we can write the acceleration as,
%
\begin{IEEEeqnarray}{rCl}
\mathbf{a}(t) &=& \dot{\mathbf{v}}(t) \nonumber \\
              &=& \frac{d}{dt}\left( \dot{s}(t) \mathbf{e}_{T}(t) \right) \nonumber \\
              &=& \ddot{s}(t) \mathbf{e}_{T}(t) + \dot{s}(t) \dot{\mathbf{e}}_{T}(t) \nonumber \\
              &=& \underbrace{\ddot{s}(t)}_{a_T} \mathbf{e}_{T}(t) + \underbrace{\dot{s}(t) \dot{\psi}(t)}_{a_N} \mathbf{e}_{N}(t)     ,
\end{IEEEeqnarray}



The principle upon which the discretisation is based is that $a_T$ and $a_N$ are constant for the duration of a ``manoeuvre'',
%
\begin{IEEEeqnarray}{rCl}
 \ddot{s}(t) &=& a_T \nonumber \\
 \dot{s}(t) \dot{\psi}(t) &=& a_N     .
\end{IEEEeqnarray}

Alone, this is not sufficient to render the model analytically solvable, but lets see how far we can get. First the speed,
%
\begin{IEEEeqnarray}{rCl}
 \dot{s}(t) &=& s_0 + a_T t     ,
\end{IEEEeqnarray}
 
\noindent where $s_0 = s(0)$ is the fixed intial value. Second the rotation,
%
\begin{IEEEeqnarray}{rCl}
 \dot{\psi}(t) &=& \frac{a_N}{ \dot{s}(t) } \nonumber \\
               &=& \frac{a_N}{ s_0 + a_T t } \nonumber \\
 \Delta \psi(t)   &=& \frac{a_N}{a_T} \log \left[ \frac{s(t)}{s_0} \right]     .
\end{IEEEeqnarray}

The variable $\Delta \psi(t)$ is the total rotation since the start of the manoeuvre. In general, there is no reason for this rotation to lie within a plane, so $\Delta \psi(t)$ is \emph{not} the same as the angle between $\mathbf{e}_{T}(0)$ and $\mathbf{e}_{T}(t)$.

Now we group the unit vector differential equations together into a matrix equation,
%
\begin{IEEEeqnarray}{rCl}
\underbrace{\frac{d}{dt}\begin{bmatrix}\mathbf{e}_T(t)^T \\ \mathbf{e}_N(t)^T \\ \mathbf{e}_B(t)^T \end{bmatrix}}_{\dot{E}(t)} &=& \underbrace{\begin{bmatrix}0 & \dot{\psi}(t) & 0 \\ - \dot{\psi}(t) & 0 & - \dot{\phi}(t) \\ 0 & \dot{\phi}(t) & 0 \end{bmatrix}}_{F(t)} \underbrace{\begin{bmatrix}\mathbf{e}_{T}^T(t) \\ \mathbf{e}_{N}^T(t) \\ \mathbf{e}_{B}^T(t) \end{bmatrix}}_{E(t)}      .
\end{IEEEeqnarray}

This can then be solved by comparison with the following matrix exponential identity,
%
\begin{IEEEeqnarray}{rCl}
 A(t) &=& \exp \left(B(t)\right) \nonumber \\
 \dot{A}(t) &=& \dot{B}(t) \exp \left(B(t)\right) \nonumber \\
 \dot{A}(t) &=& \dot{B}(t) A(t)
\end{IEEEeqnarray}

Thus, the solution is,
%
\begin{IEEEeqnarray}{rCl}
 E(t) &=& \exp \left( \int_0^t F(\tau) d\tau + K \right) \nonumber \\
      &=& \underbrace{\exp \left( \int_0^t F(\tau) d\tau \right)}_{A(t)} E(0)    ,
\end{IEEEeqnarray}

\noindent where $E_0 = E(0)$ and the second line follows from the intial conditions. We don't know what it is, but for now define a variable for the change in roll, 
%
\begin{IEEEeqnarray}{rCl}
 \Delta \phi(t) &=& \int_0^t \dot{\phi}(\tau) d\tau
\end{IEEEeqnarray}

The matrix integral can now be written as,
%
\begin{IEEEeqnarray}{rCl}
 M(t) &=& \int_0^t F(\tau) d\tau \nonumber \\
      &=& \begin{bmatrix}0 & \Delta \psi(t) & 0 \\ - \Delta \psi(t) & 0 & - \Delta \phi(t) \\ 0 & \Delta \phi(t) & 0 \end{bmatrix}     .
\end{IEEEeqnarray}

To summarise,
%
\begin{IEEEeqnarray}{rCl}
 E(t) &=& A(t) E(t) \nonumber \\
 A(t) &=& \exp\left(M(t)\right) \nonumber \\
 M(t) &=& \begin{bmatrix}0 & \Delta \psi(t) & 0 \\ - \Delta \psi(t) & 0 & - \Delta \phi(t) \\ 0 & \Delta \phi(t) & 0 \end{bmatrix} \nonumber \\
 \Delta \psi(t) &=& \frac{a_N}{a_T} \log \left[ \frac{s(t)}{s_0} \right] \nonumber \\
 \Delta \phi(t) &=& \int_0^t \dot{\phi}(\tau) d\tau \nonumber \\
 \dot{s}(t) &=& s_0 + a_T t \nonumber
\end{IEEEeqnarray}






\section{Planar Motion}





\section{Constant Roll}

\end{document}
